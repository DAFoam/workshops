\documentclass{bredelebeamer}
\usepackage{graphicx}
\usepackage{amsmath,amsfonts,amsthm}
\usepackage{hyperref}
\usepackage{setspace}
%\usepackage{media9} % for movies
%\usepackage{multimedia} % for movies
\usepackage{caption} % for figure caption
\usepackage{animate}
\usepackage[final]{listings} % for listing source code
\usefonttheme[onlymath]{serif} % make the equations look better
% define box for listing
\definecolor{myblue}{HTML}{4C72B0}
\definecolor{myred}{HTML}{C54E52}
\definecolor{mygreen}{HTML}{56A968}
\lstset{
      language=C++,
      basicstyle=\ttfamily,
      frame=single,
      columns=flexible,
      breaklines=true,
      commentstyle=\color{mygreen},
      keywordstyle = \color{myblue},
      stringstyle  = \color{orange},
      }
\usepackage[absolute,overlay]{textpos} 
\setlength{\parskip}{0.9em}
\setbeamertemplate{navigation symbols}{}
\newcommand*\dif{\mathop{}\!\mathrm{d}}
%%%%%%%%% some settings
\captionsetup[figure]{labelformat=empty} % remove the prefix 'figure' for figure caption
\captionsetup[table]{labelformat=empty} % remove the prefix 'table' for figure caption
\setbeamertemplate{headline}{} %%%%%%%%%%% remove header %%%%%%%%%%%
\setbeamerfont{frametitle}{size=\Large} %% title font size

%%%%%%%%%% title
\title[ ]{DAFoam Workshop 2022}
\subtitle{v3.0.0}

\author{Ping He and Bernardo Pacini \\ ~ \\June 8, 2022 }


\date[June 8, 2022]{}

\setbeamercolor{title}{fg=Black,bg=White!0}
\setbeamercolor{frametitle}{fg=Black,bg=White!0}
%gets rid of footer
%will override 'frame number' instruction above
%comment out to revert to previous/default definitions
\setbeamertemplate{footline}[page number]

\begin{document}

%---------------------------------------------------------------%
\begin{frame}
  \titlepage
\end{frame}

%---------------------------------------------------------------%


%---------------------------------------------------------------%
\begin{frame}{Objectives}

After this workshop, you should be able to
\begin{itemize}
  \setlength\itemsep{1em}
 \item Get familiar with the new features and interfaces in DAFoam v3
 \item Run aerodynamic \& aerostructural optimizations with DAFoam v3
 \item Modify/add DAFoam's C++ and Python codes for a new feature
\end{itemize}

\end{frame}
%---------------------------------------------------------------%

%---------------------------------------------------------------%
\begin{frame}{A few notes}

  \begin{itemize}
    \setlength\itemsep{1em}
   \item We assume you are familiar with DAFoam v2.
   \item This workshop has \textbf{hands-on} examples.
   \item \textbf{Stop} us at any time if you have questions.
   \item The online meeting will be \textbf{recorded}.
   \item All the materials are available at \url{https://github.com/dafoam/workshops}.
\end{itemize}
  
\end{frame}
%---------------------------------------------------------------%

%---------------------------------------------------------------%
\begin{frame}{Outline}
  \tableofcontents
\end{frame}
%---------------------------------------------------------------%

% ************************************************************************************
\section{DAFoam v3 Introduction}
\renewcommand{\arraystretch}{2}

%---------------------------------------------------------------%
\begin{frame}{}
  \center \Large DAFoam v3 Introduction
\end{frame}
%---------------------------------------------------------------%

%---------------------------------------------------------------%
\begin{frame}{What is DAFoam?}

  {\large DAFoam: \textbf{D}iscrete \textbf{A}djoint with Open\textbf{FOAM}}

  ~

  DAFoam develops an efficient discrete adjoint method to perform high-fidelity multidisciplinary design optimization. DAFoam has the following features:
  \center \normalsize
  \begin{itemize}
    \setlength\itemsep{1em}
    \item It uses a popular open-source package OpenFOAM (\url{www.openfoam.com}) for multiphysics analysis
    \item It implements a Jacobian-free discrete adjoint approach with competitive speed, scalability, and accuracy
    \item It has a convenient Python interface to couple with OpenMDAO (\url{www.openmdao.org}) for multidisciplinary design optimization
  \end{itemize}

\end{frame}
%---------------------------------------------------------------%

%---------------------------------------------------------------%
\begin{frame}{What is new in DAFoam v3?}
  DAFoam v3 is a major release that integrated DAFoam and OpenMDAO for multidisciplinary design optimization (MDO) through the OpenMDAO/Mphys interface

  \begin{itemize}
    \setlength\itemsep{1em}
    \item It developed a new Python interface (mphys/mphys\_dafoam.py) to Mphys and OpenMDAO for MDO
    \item Most of the settings are same as v2, but DAFoam v3 uses very different runScript.py because it is coupled with OpenMDAO.
    \item You need to update dependency versions for MDO in v3. Check the DAFoam website (\url{https://dafoam.github.io}).
    \item DAFoam v3 is compatible with all v2 run scripts.
  \end{itemize}
  
\end{frame}
%---------------------------------------------------------------%


% ************************************************************************************
\section{An OpenMDAO tutorial}
\renewcommand{\arraystretch}{2}

%---------------------------------------------------------------%
\begin{frame}{}
  \center \Large An OpenMDAO tutorial

  \normalsize
  Before using DAFoam v3, we need to learn how OpenMDAO works.
  Refer to OpenMDAO's documentation for more advanced usage. \\ 
  \url{https://openmdao.org/newdocs/versions/latest/main.html}
\end{frame}
%---------------------------------------------------------------%

%---------------------------------------------------------------%
\begin{frame}[fragile]{Optimizing a two-component system}
  \begin{figure}
    \includegraphics[width=\linewidth]{images/example_xdsm.pdf} 
    \caption{Design structure diagram for the two-component system. The red component is implicit and the green one is explicit. The design variable is $x$ and the objective function is $f$. $y$ is the solution from the implicit component and is passed to the explicit component as the input to compute $f$.}
  \end{figure}
\end{frame}
%---------------------------------------------------------------%


%---------------------------------------------------------------%
\begin{frame}[fragile]{Download DAFoam Docker image and examples}

  The most general way to run the above case is to use the DAFoam Docker image, which has OpenMDAO installed already.

  First, install Docker following this website: \\ \vspace{0.1in} \small \url{https://dafoam.github.io/mydoc_get_started_download_docker.html} \normalsize

  Once done, verify the installation by running:
  \footnotesize
  \lstset{ language=bash }
  \begin{lstlisting}
docker --version
  \end{lstlisting}
  \normalsize
     
  Then run this command to download the DAFoam Docker image: 

  \footnotesize
  \lstset{ language=bash }
  \begin{lstlisting}
docker pull dafoam/opt-packages:v3.0.0
  \end{lstlisting}
  \normalsize
  
  Finally, download the workshop examples at: \\ \vspace{0.1in}
  \small \texttt{\url{https://github.com/dafoam/workshops}}

\end{frame}
%---------------------------------------------------------------%


%---------------------------------------------------------------%
\begin{frame}[fragile]{Start a Docker container}

  If you use Linux or MacOS, open a terminal and use the \texttt{cd} command to go this folder on your local computer. If you put the workshops folder in the \$HOME directory, the command may look like:

  \footnotesize
  \begin{lstlisting}
cd $HOME/workshops/2022_Summer/examples/openmdao
  \end{lstlisting}
  \normalsize
  
  Then, run this command to start a Docker container:

  \footnotesize
  \begin{lstlisting}
docker run -it --rm -u dafoamuser --mount \
"type=bind,src=$(pwd),target=/home/dafoamuser/mount" \ 
-w /home/dafoamuser/mount dafoam/opt-packages:v3.0.0 bash
  \end{lstlisting}
  \normalsize

  If you use Windows, open the Prompt Command terminal, use the \texttt{cd} command to go to the above folder, and run this command: \\ \vspace{0.1in}

  \footnotesize
  \begin{lstlisting}
docker run -it --rm -u dafoamuser --mount \
"type=bind,src=%cd%,target=/home/dafoamuser/mount" \
-w /home/dafoamuser/mount dafoam/opt-packages:v3.0.0 bash
  \end{lstlisting}
  \normalsize

  Once in a Docker container, you should see something like:
  \footnotesize
  \lstset{ language=bash }
  \begin{lstlisting}
dafoamuser@cddb89839078:~/mount$ 
  \end{lstlisting}
  \normalsize

\end{frame}
%---------------------------------------------------------------%



%---------------------------------------------------------------%
\begin{frame}[fragile]{Run the case}

  Once in the docker container, use the \texttt{ls} command to check if you are in the correct directory. You should see something like this:

  \footnotesize
  \begin{lstlisting}
dafoamuser@cddb89839078:~/mount$ ls
runScript.py
  \end{lstlisting}
  \normalsize

  Finally, you can run the case with this command:

  \footnotesize
  \begin{lstlisting}
python runScript.py
  \end{lstlisting}
  \normalsize

  Expected output:

  \footnotesize
  \begin{lstlisting}
dafoamuser@cddb89839078:~/mount$ python runScript.py 
Optimization terminated successfully    (Exit mode 0)
            Current function value: [0.87500003]
            Iterations: 11
            Function evaluations: 11
            Gradient evaluations: 11
Optimization Complete
-----------------------------------
f_opt: [0.87500003]
x_opt: [5.54015542]
  \end{lstlisting}
  \normalsize

\end{frame}
%---------------------------------------------------------------%


%---------------------------------------------------------------%
\begin{frame}[fragile]{An alternative option without Docker}

If you already have a Python 3 on your computer, you can directly run the OpenMDAO case without using the Docker. First, load your Python 3 environment, and run this command in your terminal to install OpenMDAO
  \footnotesize
  \lstset{ language=bash }
  \begin{lstlisting}
pip install openmdao==3.16.0
  \end{lstlisting}
  \normalsize
     
  Then go to the tutorial folder:

  \footnotesize
  \begin{lstlisting}
cd $HOME/workshops/2022_Summer/examples/openmdao
  \end{lstlisting}
  \normalsize

  And run the case:

  \footnotesize
  \lstset{ language=bash }
  \begin{lstlisting}
python runScript.py
  \end{lstlisting}
  \normalsize

\end{frame}
%---------------------------------------------------------------%

%---------------------------------------------------------------%
\begin{frame}[fragile]{N2 diagram}
  After the case is finished, you should see a N2 diagram (n2.html) generated for this case. The runScript.py file essential sets the components and their data transfer (connection) for the optimization.
  \begin{figure}
    \includegraphics[width=0.8\linewidth]{images/example_n2.pdf} 
    \caption{The N2 diagram for the two-component optimization.}
  \end{figure}
\end{frame}
%---------------------------------------------------------------%


%---------------------------------------------------------------%
\begin{frame}[fragile]{runScript.py details (1/3)}
  \footnotesize
  \lstset{ language=python }
  \begin{lstlisting}
class ImplicitEqn(om.ImplicitComponent):
    def setup(self):
        # define input
        self.add_input("x", val=1.0)
        # define output
        self.add_output("y", val=1.0)

    def setup_partials(self):
        # Finite difference all partials.
        self.declare_partials("*", "*", method="fd")

    def apply_nonlinear(self, inputs, outputs, residuals):
        # get the input and output and compute the residual
        # R = e^(-x * y) - y
        # NOTE: we use [0] here because OpenMDAO assumes all inputs
        # and outputs are arrays. If the input is a scalar, OpenMDAO
        # will create an array that has size 1, so to get its value
        # we have to use [0]
        x = inputs["x"][0]
        y = outputs["y"][0]
        residuals["y"] = np.exp(-x * y) - y
  \end{lstlisting}
  \normalsize
\end{frame}
%---------------------------------------------------------------%

%---------------------------------------------------------------%
\begin{frame}[fragile]{runScript.py details (2/3)}
  \footnotesize
  \lstset{ language=python }
  \begin{lstlisting}
class ImplicitEqn(om.ImplicitComponent):
    def setup(self):
        # define input
        self.add_input("x", val=1.0)
        # define output
        self.add_output("y", val=1.0)

    def setup_partials(self):
        # Finite difference all partials.
        self.declare_partials("*", "*", method="fd")

    def apply_nonlinear(self, inputs, outputs, residuals):
        # get the input and output and compute the residual
        # R = e^(-x * y) - y
        # NOTE: we use [0] here because OpenMDAO assumes all inputs
        # and outputs are arrays. If the input is a scalar, OpenMDAO
        # will create an array that has size 1, so to get its value
        # we have to use [0]
        x = inputs["x"][0]
        y = outputs["y"][0]
        residuals["y"] = np.exp(-x * y) - y
  \end{lstlisting}
  \normalsize
\end{frame}
%---------------------------------------------------------------%


%---------------------------------------------------------------%
\begin{frame}[fragile]{runScript.py details (3/3)}
  \footnotesize
  \lstset{ language=python }
  \begin{lstlisting}
# create an OpenMDAO problem object
prob = om.Problem()
# now add the implicit component defined above to prob
prob.model.add_subsystem("ImplicitEqn", ImplicitEqn(), promotes=["*"])
# add the objective explicit component defined above to prob
prob.model.add_subsystem("Objective", Objective(), promotes=["*"],)
# set the linear/nonlinear equation solution for the implicit component
prob.model.nonlinear_solver = om.NewtonSolver(solve_subsystems=False)
prob.model.linear_solver = om.ScipyKrylov()
# set the design variable and objective function
prob.model.add_design_var("x", lower=-10, upper=10)
prob.model.add_objective("f", scaler=1)
# setup the optimizer
prob.driver = om.ScipyOptimizeDriver()
prob.driver.options["optimizer"] = "SLSQP"
# setup the problem
prob.setup()
# write the n2 diagram
om.n2(prob, show_browser=False, outfile="n2.html")
# run the optimization
prob.run_driver()
  \end{lstlisting}
  \normalsize
\end{frame}
%---------------------------------------------------------------%

%---------------------------------------------------------------%
\begin{frame}[fragile]{Summary}
  \begin{itemize}
    \setlength\itemsep{1em}
   \item To use OpenMDAO for optimizations, one needs to write classes for each component in the system, define their inputs and outputs, and implement the way to compute the outputs (explicit components) or residuals (implicit components). 
   \item Then one needs to add these components to the optimization problem in the run script, set their connection, set the design variables, objective and constraint functions, and run the optimization.
   \item A new open-source interface Mphys was recently developed (\url{https://github.com/openmdao/mphys}) to rewrite the MACH-Aero modules' interfaces (e.g., pyGeo, IDWarp) into the OpenMDAO component format. 
   \item DAFoam v3 has a Python interface to OpenMDAO/Mphys (dafoam/mphys/mphys\_dafoam.py). So to run optimizations with DAFoam v3, one needs to use the new run script that sets the components, data transfer, design variables, objective and constraint functions in the optimization (see next section).
  \end{itemize}
\end{frame}
%---------------------------------------------------------------%


% ************************************************************************************
\section{Aerodynamic optimization}
\renewcommand{\arraystretch}{2}

%---------------------------------------------------------------%
\begin{frame}{}
  \center \Large Aerodynamic optimization
\end{frame}
%---------------------------------------------------------------%



%---------------------------------------------------------------%
\begin{frame}{Summary of the NACA0012 subsonic case}

  \begin{table}
    \renewcommand{\arraystretch}{1.5}
    \small
    \centering
    \label{tab:implemented_models}
    \begin{tabular}{llllllllllll}
    \hline
    Optimizer   & IPOPT \\
    Flow and adjoint solvers  & DARhoSimpleFoam  \\
    Geometry  & NACA0012 \\
    Mesh  & 4\,032 cells\\
    Objective function  & $C_d$ \\
    Design variables & 20 FFDs and $\alpha$ \\
    Constraint & $C_l=0.5$, thickness, volume, TE/LE \\
    $U_\infty$  & 100 m/s \\
    $Re$  & 6.7$\times10^6$\\
    Turbulence Model  & Spalart--Allmaras\\
     \hline
    \end{tabular}
  \end{table}

  You can find the case settings in workshops/2022\_Summer/examples/naca0012.
\end{frame}
%---------------------------------------------------------------%


%---------------------------------------------------------------%
\begin{frame}[fragile]{Run the case}

  First, use the \texttt{cd} command to go the workshops/2022\_Summer/examples/naca10012 folder. Then, use the command in Slide 11 to start a Docker container. Next, use the \texttt{ls} command to check if you are in the correct directory.
  \footnotesize
  \lstset{ language=bash }
  \begin{lstlisting}
dafoamuser@bd114f3f7c94:~/mount$ ls
0.orig       FFD       genAirFoilMesh.py  preProcessing.sh  runScript.py
Allclean.sh  constant  paraview.foam      profiles          system
  \end{lstlisting}
  \normalsize

  Next, run this command to generate the mesh:
  \footnotesize
  \lstset{ language=bash }
  \begin{lstlisting}
./preProcessing.sh
  \end{lstlisting}
  \normalsize

  Finally, run the optimization with 2 cores.
  \footnotesize
  \lstset{ language=bash }
  \begin{lstlisting}
mpirun -np 2 python runScript.py | tee 2>&1 logOpt.txt
  \end{lstlisting}
  \normalsize

  The optimization log will be printed to the screen and saved to \texttt{logOpt.txt}. In addition, the optimizer will write a separate log to the disk \texttt{opt\_IPOPT.txt}.

\end{frame}
%---------------------------------------------------------------%

%---------------------------------------------------------------%
\begin{frame}[fragile]{How to post-process the optimization results?}

  This has been covered in the 2021 workshop, refer to Slides 29 to 47 from 
  \url{https://github.com/DAFoam/workshops/blob/main/2021_Summer/slides/2021_Summer_Workshop.pdf}

\end{frame}
%---------------------------------------------------------------%


%---------------------------------------------------------------%
\begin{frame}[fragile]{NACA0012 N2 diagram}

  \begin{figure}
    \includegraphics[width=\linewidth]{images/naca0012_n2.pdf} 
    \caption{The N2 diagram for the NACA0012 aerodynamic optimization.}
  \end{figure}

\end{frame}
%---------------------------------------------------------------%


%---------------------------------------------------------------%
\begin{frame}[fragile]{runScript.py details (1/10)}
Most of the settings in DAFoam v3 are same as v2. The main difference is that DAFoam v3 uses a very different runScript.py
  \footnotesize
  \lstset{ language=bash }
  \begin{lstlisting}
# import modules
import os
import argparse
import numpy as np
from mpi4py import MPI
import openmdao.api as om
from mphys.multipoint import Multipoint
from dafoam.mphys import DAFoamBuilder, OptFuncs
from mphys.scenario_aerodynamic import ScenarioAerodynamic
from pygeo.mphys  import OM_DVGEOCOMP
from pygeo import geo_utils
    
# input arguments for runScript.py    
parser = argparse.ArgumentParser()
# which optimizer to use. Options are: IPOPT (default), SLSQP, and SNOPT
parser.add_argument("-optimizer", help="optimizer to use", type=str, default="IPOPT")
# which task to run. Options are: opt (default), runPrimal, runAdjoint, checkTotals
parser.add_argument("-task", help="type of run to do", type=str, default="opt")
args = parser.parse_args()
  \end{lstlisting}
  \normalsize

\end{frame}
%---------------------------------------------------------------%

%---------------------------------------------------------------%
\begin{frame}[fragile]{runScript.py details (2/10)}
  The input parameters, daOptions, meshOptions are same as DAFoam v2. Next, we need to set a Multipoint Top class. We need to add all the components in the \texttt{setup} function.
  \footnotesize
  \lstset{ language=bash }
  \begin{lstlisting}
# Top class to setup the optimization problem
class Top(Multipoint):
    def setup(self):
        # create the builder to initialize the DASolvers
        dafoam_builder = DAFoamBuilder(daOptions, meshOptions, scenario="aerodynamic")
        dafoam_builder.initialize(self.comm)
        # add the design variable component 
        self.add_subsystem("dvs", om.IndepVarComp(), promotes=["*"])
        # add the mesh component
        self.add_subsystem("mesh", dafoam_builder.get_mesh_coordinate_subsystem())
        # add the geometry component (FFD)
        self.add_subsystem("geometry", OM_DVGEOCOMP(ffd_file="FFD/wingFFD.xyz"))
        # add a scenario and pass the builder to it
        self.mphys_add_scenario("cruise", ScenarioAerodynamic(aero_builder=dafoam_builder))
        # need to do manually connection
        self.connect("mesh.x_aero0", "geometry.x_aero_in")
        self.connect("geometry.x_aero0", "cruise.x_aero")
  \end{lstlisting}
  \normalsize
  \end{frame}
%---------------------------------------------------------------%

%---------------------------------------------------------------%
\begin{frame}[fragile]{runScript.py details (3/10)}
Next, we need to do the proper connection and configurations for the above components in the \texttt{configure} function. We also need to set the design variables, objective and constraint functions. The overall process is similar to v2 but it needs to use a different syntax to set up.
    \footnotesize
    \lstset{ language=bash }
    \begin{lstlisting}
def configure(self):
    # configure and setup perform a similar function, i.e., initialize 
    # the optimization. The configure will be run after setup
    # add the objective function to the cruise scenario
    self.cruise.aero_post.mphys_add_funcs()
    # get the surface coordinates from the mesh component
    points = self.mesh.mphys_get_surface_mesh()
    # add pointset to the geometry component
    self.geometry.nom_add_discipline_coords("aero", points)
    # set the triangular points to the geo component for geo constraints
    tri_points = self.mesh.mphys_get_triangulated_surface()
    self.geometry.nom_setConstraintSurface(tri_points)
  \end{lstlisting}
  \normalsize
  \end{frame}
%---------------------------------------------------------------%

%---------------------------------------------------------------%
\begin{frame}[fragile]{runScript.py details (4/10)}
Next, we define an angle of attack (aoa) function to change the far field velocity, and add it to the "cruise" scenario as the design variable.
  \footnotesize
  \lstset{ language=bash }
  \begin{lstlisting}
# define an angle of attack function to change the U direction at the far field
def aoa(val, DASolver):
    aoa = val[0] * np.pi / 180.0
    U = [float(U0 * np.cos(aoa)), float(U0 * np.sin(aoa)), 0]
    # we need to update the U value only
    DASolver.setOption("primalBC", {"U0": {"value": U}})
    DASolver.updateDAOption()

# pass this aoa function to the cruise group
self.cruise.coupling.solver.add_dv_func("aoa", aoa)
self.cruise.aero_post.add_dv_func("aoa", aoa)
  \end{lstlisting}
  \normalsize
  \end{frame}
%---------------------------------------------------------------%

%---------------------------------------------------------------%
\begin{frame}[fragile]{runScript.py details (5/10)}
We can then add the shape as the design variable by choosing all the FFD points. The setup is similar to v2.
  \footnotesize
  \lstset{ language=bash }
  \begin{lstlisting}
# select the FFD points to move
pts = self.geometry.DVGeo.getLocalIndex(0)
indexList = pts[:, :, :].flatten()
PS = geo_utils.PointSelect("list", indexList)
nShapes = self.geometry.nom_addGeoDVLocal(dvName="shape", pointSelect=PS)
  \end{lstlisting}
  \normalsize
  \end{frame}
%---------------------------------------------------------------%

%---------------------------------------------------------------%
\begin{frame}[fragile]{runScript.py details (6/10)}
Because it is a symmetric case, we need to impose the symmetry constraint for the k=0 and k=1 level FFD points. The setup is very similar to v2.
  \footnotesize
  \lstset{ language=bash }
  \begin{lstlisting}
# setup the symmetry constraint to link the y displacement between k=0 and k=1
nFFDs_x = pts.shape[0]
nFFDs_y = pts.shape[1]
indSetA = []
indSetB = []
for i in range(nFFDs_x):
    for j in range(nFFDs_y):
        indSetA.append(pts[i, j, 0])
        indSetB.append(pts[i, j, 1])
self.geometry.nom_addLinearConstraintsShape("linearcon", indSetA, indSetB, factorA=1.0, factorB=-1.0)
  \end{lstlisting}
  \normalsize
  \end{frame}
%---------------------------------------------------------------%

%---------------------------------------------------------------%
\begin{frame}[fragile]{runScript.py details (7/10)}
Similar to v2, we need to set leList and teList for the volume and thickness constraint. We also use the LETE constraint to fix the leading edge and trailing edge.
  \footnotesize
  \lstset{ language=bash }
  \begin{lstlisting}
# setup the volume and thickness constraints
leList = [[1e-4, 0.0, 1e-4], [1e-4, 0.0, 0.1 - 1e-4]]
teList = [[0.998 - 1e-4, 0.0, 1e-4], [0.998 - 1e-4, 0.0, 0.1 - 1e-4]]
self.geometry.nom_addThicknessConstraints2D("thickcon", leList, teList, nSpan=2, nChord=10)
self.geometry.nom_addVolumeConstraint("volcon", leList, teList, nSpan=2, nChord=10)
# add the LE/TE constraints
self.geometry.nom_add_LETEConstraint("lecon", volID=0, faceID="iLow", topID="k")
self.geometry.nom_add_LETEConstraint("tecon", volID=0, faceID="iHigh", topID="k")
  \end{lstlisting}
  \normalsize
  \end{frame}
%---------------------------------------------------------------%

%---------------------------------------------------------------%
\begin{frame}[fragile]{runScript.py details (8/10)}
Now, we can add the shape and aoa variables as the output for the "dvs" component and use them as the design variables. Note that we need to manually connect dvs' output to the cruise and geometry component. See the N2 diagram.
  \footnotesize
  \lstset{ language=bash }
  \begin{lstlisting}
# add the design variables to the dvs component's output
self.dvs.add_output("shape", val=np.array([0] * nShapes))
self.dvs.add_output("aoa", val=np.array([aoa0]))
# manually connect the dvs output to the geometry and cruise
self.connect("aoa", "cruise.aoa")
self.connect("shape", "geometry.shape")

# define the design variables to the top level
self.add_design_var("shape", lower=-1.0, upper=1.0, scaler=1.0)
self.add_design_var("aoa", lower=0.0, upper=10.0, scaler=1.0)
  \end{lstlisting}
  \normalsize
  \end{frame}
%---------------------------------------------------------------%

%---------------------------------------------------------------%
\begin{frame}[fragile]{runScript.py details (9/10)}
Finally, we add the objective and constraint functions.
  \footnotesize
  \lstset{ language=bash }
  \begin{lstlisting}
# add objective and constraints to the top level
self.add_objective("cruise.aero_post.CD", scaler=1.0)
self.add_constraint("cruise.aero_post.CL", equals=CL_target, scaler=1.0)
self.add_constraint("geometry.thickcon", lower=0.5, upper=3.0, scaler=1.0)
self.add_constraint("geometry.volcon", lower=1.0, scaler=1.0)
self.add_constraint("geometry.tecon", equals=0.0, scaler=1.0, linear=True)
self.add_constraint("geometry.lecon", equals=0.0, scaler=1.0, linear=True)
self.add_constraint("geometry.linearcon", equals=0.0, scaler=1.0, linear=True)
  \end{lstlisting}
  \normalsize
  \end{frame}
%---------------------------------------------------------------%

%---------------------------------------------------------------%
\begin{frame}[fragile]{runScript.py details (10/10)}
Once the Top class is created, we pass it as the OpenMDAO problem's model. Then, we can run the OpenMDAO's \texttt{driver} for optimization.
  \footnotesize
  \lstset{ language=bash }
  \begin{lstlisting}
prob = om.Problem()
prob.model = Top()
prob.setup(mode="rev") # reverse mode AD
om.n2(prob, show_browser=False, outfile="mphys.html")
# initialize the optimization function
optFuncs = OptFuncs(daOptions, prob)
# use pyOptSparse to run optimization
prob.driver = om.pyOptSparseDriver()
prob.driver.options["optimizer"] = args.optimizer
# ........... pyOptSparse optimizer setup .........
# set the output option
prob.driver.options["debug_print"] = ["nl_cons", "objs", "desvars"]
# prob.driver.options["print_opt_prob"] = True
prob.driver.hist_file = "OptView.hst"
# select task to run
if args.task == "opt":
    # solve CL by changing aoa
    optFuncs.findFeasibleDesign(["cruise.aero_post.CL"], ["aoa"], targets=[CL_target])
    # run the optimization
    prob.run_driver()
  \end{lstlisting}
  \normalsize
  \end{frame}
%---------------------------------------------------------------%

%---------------------------------------------------------------%
\begin{frame}[fragile]{Summary}
    \begin{itemize}
      \setlength\itemsep{1em}
      \item The runScript.py is essentially an OpenMDAO run script. So we suggest you first learn how OpenMDAO works by going through the OpenMDAO's documentation (\url{https://openmdao.org/newdocs/versions/latest/main.html}).
     \item We can use the above script to run any airfoil aerodynamic optimization with DAFoam v3. If you want to change the flight conditions, FFD points, airfoil profiles, refer to the DAFoam FAQ. \url{https://dafoam.github.io/mydoc_get_started_faq.html}. Note that these changes are for v2 but they also work for v3.
     \item For 3D wing aerodynamic optimization, refer to the run script \url{https://github.com/DAFoam/tutorials/blob/main/MACH_Tutorial_Wing/runScript_Aero.py}
     \item For multipoint optimization, refer to the run script \url{https://github.com/DAFoam/tutorials/blob/main/NACA0012_Airfoil/multipoint/runScript.py}
    \end{itemize}
  \end{frame}
%---------------------------------------------------------------%
  

% ************************************************************************************
\section{Aerostructural optimization}
\renewcommand{\arraystretch}{2}

%---------------------------------------------------------------%
\begin{frame}{}
  \center \Large Aerostructural optimization
\end{frame}
%---------------------------------------------------------------%


% ************************************************************************************
\section{Add new features to DAFoam}
\renewcommand{\arraystretch}{2}

%---------------------------------------------------------------%
\begin{frame}{}
  \center \Large Add new features to DAFoam
  \normalsize
  \begin{itemize}
    \setlength\itemsep{1em}
    \item DAFoam source code structure
    \item text
  \end{itemize}
\end{frame}
%---------------------------------------------------------------%

%---------------------------------------------------------------%
\begin{frame}{}
  \center \Large DAFoam source code structure

  \large There are three code layers:  \\ 
  OpenFOAM, Cython, and Python
\end{frame}
%---------------------------------------------------------------%

%---------------------------------------------------------------%
\begin{frame}[fragile]{OpenFOAM layer}
  The C++ layer contains classes for the actual adjoint computation and is written based on OpenFOAM. The C++ source codes are in dafoam/src/adjoint.
  \begin{itemize}
    \setlength\itemsep{1em}
    \item When DAFoam is compiled, it will generate dynamic libraries for three different conditions: incompressible, compressible, and solid. In other words, it will compile three .so files. Refer to dafoam/src/adjoint/Make/files\_Incompressible for which files are compiled for the incompressible case.
  \end{itemize}
\end{frame}
%---------------------------------------------------------------%

%---------------------------------------------------------------%
\begin{frame}[plain]{}
  \Huge \centering
  Thank you!
\end{frame}
%---------------------------------------------------------------%

\end{document}
