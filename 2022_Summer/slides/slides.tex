\documentclass{bredelebeamer}
\usepackage{graphicx}
\usepackage{amsmath,amsfonts,amsthm}
\usepackage{hyperref}
\usepackage{setspace}
%\usepackage{media9} % for movies
%\usepackage{multimedia} % for movies
\usepackage{caption} % for figure caption
\usepackage{animate}
\usepackage[final]{listings} % for listing source code
\usefonttheme[onlymath]{serif} % make the equations look better
% define box for listing
\definecolor{myblue}{HTML}{4C72B0}
\definecolor{myred}{HTML}{C54E52}
\definecolor{mygreen}{HTML}{56A968}
\lstset{
      language=C++,
      basicstyle=\ttfamily,
      frame=single,
      columns=flexible,
      breaklines=true,
      commentstyle=\color{mygreen},
      keywordstyle = \color{myblue},
      stringstyle  = \color{orange},
      }
\usepackage[absolute,overlay]{textpos} 
\setlength{\parskip}{0.9em}
\setbeamertemplate{navigation symbols}{}
\newcommand*\dif{\mathop{}\!\mathrm{d}}
%%%%%%%%% some settings
\captionsetup[figure]{labelformat=empty} % remove the prefix 'figure' for figure caption
\captionsetup[table]{labelformat=empty} % remove the prefix 'table' for figure caption
\setbeamertemplate{headline}{} %%%%%%%%%%% remove header %%%%%%%%%%%
\setbeamerfont{frametitle}{size=\Large} %% title font size

%%%%%%%%%% title
\title[ ]{DAFoam Workshop 2022}
\subtitle{v3.0.0}

\author{Ping He and Bernardo Pacini \\ ~ \\June 8, 2022 }


\date[June 8, 2022]{}

\setbeamercolor{title}{fg=Black,bg=White!0}
\setbeamercolor{frametitle}{fg=Black,bg=White!0}
%gets rid of footer
%will override 'frame number' instruction above
%comment out to revert to previous/default definitions
\setbeamertemplate{footline}[page number]

\begin{document}

%---------------------------------------------------------------%
\begin{frame}
  \titlepage
\end{frame}

%---------------------------------------------------------------%


%---------------------------------------------------------------%
\begin{frame}{Objectives}

After this workshop, you should be able to
\begin{itemize}
  \setlength\itemsep{1em}
 \item Get familiar with the new features and interfaces in DAFoam v3
 \item Run aerodynamic \& aerostructural optimizations with DAFoam v3
 \item Modify/add DAFoam's C++ and Python codes for a new feature
\end{itemize}

\end{frame}
%---------------------------------------------------------------%

%---------------------------------------------------------------%
\begin{frame}{A few notes}

  \begin{itemize}
    \setlength\itemsep{1em}
   \item We assume you are familiar with DAFoam v2.
   \item This workshop has \textbf{hands-on} examples.
   \item \textbf{Stop} us at any time if you have questions.
   \item The online meeting will be \textbf{recorded}.
   \item All the materials are available at \url{https://github.com/dafoam/workshops}.
\end{itemize}
  
\end{frame}
%---------------------------------------------------------------%

%---------------------------------------------------------------%
\begin{frame}{Outline}
  \tableofcontents
\end{frame}
%---------------------------------------------------------------%

\section{DAFoam v3}
\renewcommand{\arraystretch}{2}

%---------------------------------------------------------------%
\begin{frame}{}
  \center \Large DAFoam v3 Introduction
\end{frame}
%---------------------------------------------------------------%

%---------------------------------------------------------------%
\begin{frame}{What is DAFoam?}

  {\large DAFoam: \textbf{D}iscrete \textbf{A}djoint with Open\textbf{FOAM}}

  ~

  DAFoam develops an efficient discrete adjoint method to perform high-fidelity multidisciplinary design optimization. DAFoam has the following features:
  \center \normalsize
  \begin{itemize}
    \setlength\itemsep{1em}
    \item It uses a popular open-source package OpenFOAM (\url{www.openfoam.com}) for multiphysics analysis
    \item It implements a Jacobian-free discrete adjoint approach with competitive speed, scalability, and accuracy
    \item It has a convenient Python interface to couple with OpenMDAO (\url{www.openmdao.org}) for multidisciplinary design optimization
  \end{itemize}

\end{frame}
%---------------------------------------------------------------%

%---------------------------------------------------------------%
\begin{frame}{What is new in DAFoam v3?}
  DAFoam v3 is a major release that integrated DAFoam and OpenMDAO for multidisciplinary design optimization (MDO) through the OpenMDAO/Mphys interface

  \begin{itemize}
    \setlength\itemsep{1em}
    \item It developed a new Python interface (mphys/mphys\_dafoam.py) to Mphys and OpenMDAO for MDO
    \item Most of the settings are same as v2, but DAFoam v3 uses very different runScript.py because it is coupled with OpenMDAO.
    \item You need to update dependency versions for MDO in v3. Check the DAFoam website (\url{https://dafoam.github.io}).
    \item DAFoam v3 is compatible with all v2 run scripts.
  \end{itemize}
  
\end{frame}
%---------------------------------------------------------------%

\section{DAFoam code development}
\renewcommand{\arraystretch}{2}

%---------------------------------------------------------------%
\begin{frame}{}
  \center \Large DAFoam code structure
\end{frame}
%---------------------------------------------------------------%

%---------------------------------------------------------------%
\begin{frame}[plain]{}
  \Huge \centering
  Thank you!
\end{frame}
%---------------------------------------------------------------%

\end{document}
